\documentclass[10pt,a4paper]{article}
\usepackage[utf8]{inputenc}
\usepackage{amsmath}
\usepackage{amsfonts}
\usepackage[margin=1in]{geometry}
\usepackage{amssymb}
\author{Joel Mizzoni (jmizzoni), Patrick White (ps2white)}
\begin{document}
\title{ECE358 S'16 Assignment 4}
\maketitle
\section{}
Here we answer question 1
\section{}
Here we answer question 2
\section{}
Here we answer question 3
\section{}
Here we answer question four
\section{}
Here we answer question 5
\section{}
When using the distance vector algorithm from an initial state where all path lengths are unknown, after $N$ iterations, each node will be aware of the shortest path to each node consisting of $N$ edges. This means the number of iterations it will take to converge is equal to the longest (in # of edges) shortest path in the network plus 1 (since one more iteration has to occur in order for the last nodes to update to broadcast one last message). The greatest number of edges a shortest path can have is $N - 1$, we will converge in at most $N$ iterations.
\section{}
Here we answer question 7
\end{document}
