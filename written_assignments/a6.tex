\documentclass[10pt,a4paper]{article}
\usepackage[utf8]{inputenc}
\usepackage{amsmath}
\usepackage{amsfonts}
\usepackage{booktabs}
\usepackage[margin=1in]{geometry}
\usepackage{amssymb}
\usepackage{tikz}
\author{Joel Mizzoni (jmizzoni), Patrick White (ps2white)}
\begin{document}
\title{ECE358 S'16 Assignment 6}
\maketitle
\section{}
The packet takes the path $A\rightarrow R_1 \rightarrow R_4 \rightarrow R_5 \rightarrow R_6 \rightarrow B$.
Most of the routing is trivial to explain except from $R_1$.
$R_1$ knows that $B$ is reachable both from the ASs that are adjacent to its AS, but the route to the gateway to the $R_4$ AS is shorter.
Therefore, $R_1$ forwards the packet to $R_4$, even though it results in a longer overall path than if it had routed it to $R_3$.
\section{}
In which we answer question 2
\section{}
In which we answer question three
\end{document}
