\documentclass[10pt,a4paper]{article}
\usepackage[utf8]{inputenc}
\usepackage{amsmath}
\usepackage{amsfonts}
\usepackage{booktabs}
\usepackage[margin=1in]{geometry}
\usepackage{amssymb}
\usepackage{tikz}
\author{Joel Mizzoni (jmizzoni), Patrick White (ps2white)}
\begin{document}
\title{ECE358 S'16 Assignment 6}
\maketitle
\section{}
In which we answer question 1
\section{}

X will advertise the prefixes 1.2.0.0/16 and 5.6.7.0/24 to both B and C. The reason it advertises only those prefixes to B and C is that it only wants to receieve traffic destined for hosts on X and doesn't want to be responsible for routing traffic destined for other ASes.
\section{}

\subsection{a)}
In the worst case, each node in the network forwards a packet, and each of those packets that reach a particular node. Therefore the number of packets received will be $N-1$, where $N$ is the number of nodes in the graph.

\subsection{b)}
Using RPF, each node will forward a packet to each of its neighbours at most once. Therefore a packet receives at most one packet from each of its neighbours, so the the maximum number of packets a node can receive is equal to the number of neighbours it has.

\subsection{c)}
Using a spanning tree, each packet has exactly one direct path from the source to each other node in the graph. Therefore each node will receive a packet at most once. 

\end{document}
