\documentclass[10pt,a4paper]{article}
\usepackage[utf8]{inputenc}
\usepackage{amsmath}
\usepackage{amsfonts}
\usepackage[margin=1in]{geometry}
\usepackage{amssymb}
\author{Joel Mizzoni (jmizzoni), Patrick White (ps2white)}
\begin{document}
\title{ECE358 S'16 Assignment 3}
\newcommand{\floor}[1]{\lfloor#1\rfloor}
\maketitle
\section{}
For any network arranged like this, there are $m$ subnetworks. We will begin by constructing a spanning tree for the network using $m - 1$ of the routers in the network. The minimum size of a routing table in a connected network like this is $\floor{\frac{m - 1}{2}} + 1$. The reason for this is that we assume that each router is connected to exactly 2 subnetworks.  The next hops that are stored in each routing table are those that need to be taken for each destination subnetwork on the side of the router that has the fewest subnetworks. At most this will be $\floor{\frac{m - 1}{2}}$ for a router in the center of the spanning tree. The last entry that will be stored in each table will be the prefix 0.0.0.0/0 which will match any other route not in the table. Since there are only 2 directions to travel, we know that the next hop for any packet destined for a subnet that doesn't match one of the given addresses will have to hop in the other direction. The largest table under this scheme can't be any smaller than $\floor{\frac{m - 1}{2} + 1}$ since then it would be unable to correctly route all addresses in the network without some additional topological routing scheme, which can't be established for all possible networks.

\section{}
There are several cases for this problem, most of which are trivial:
\begin{itemize}
\item any case in which $k<0$ is trivially false, since the minimum size of any set is $0$.
\item any case where $k=0$ is true if and only if $|P|=0$, since it is not possible to reduce the size of a non-empty set of prefixes by merging.
\item any case where $k\geq |P|$ is trivially true without merging, since $|P|$ is already at most $k$.
\item any case where $0<k<|P|$. This is the case that will be discussed.
\end{itemize}

There are two operations that it is trivial to show are both constant time operations - checking if two prefixes can be merged, and merging two prefixes. They are constant time because of their fixed length.

With this, it can be shown that it is possible to determine whether the problem instance is true or false in time that is polynomial in $|P|$, which we will call $n$. Consider the brute-force algorithm for merging, in which each prefix is compared to each other prefix, until either a pair is found that can be merged, or each prefix has been compared to each other, in which case the algorithm terminates. Comparing each prefix to each other takes $O(n^2)$. If a pair is found that can be merged, those two prefixes will be merged, reducing the size of the set by 1, and the algorithm will restart. Since each merge reduces the size of the set by 1, and a set of size 1 cannot be reduced further, there are at most $n-1=\Theta(n)$ iterations. Therefore, the set $P'$ is found in $O(1)*O(n^2)*\Theta(n)=O(n^3)$. It is then trivial to answer whether $|P'|\leq k$. Therefore, the question can be answered in polynomial time.

\section{}
All traffic may be routed to interface D as a result of the fact that a message to the address 12.46.129.0/5 is going to match the aggregated 12/8 entry in the routing table at point A while there will be an entry that routes directly to that address in the table at point B. A message will prefer to travel along the route with the longest matching prefix, so all traffic will be routed through provider P2 and arrive at interface D. This could be corrected by instead using its PI space address, in which case it will neither provider will get preferential treatment because neither aggregates any address that prefixes 198.134.135.0/24

\section{}

Alice is correct. 

Assume that $T$ is the minimum spanning tree of $G$, and that we are to remove node $N$ with only one incident edge $E$ on $T$. Also assume that there exists a tree $T'$ that would have lesser weight than $T - E$. If $W(T') < W(T - E)$, then $W(T' + E) < W(T - E + E) \implies W(T' + E) <  W(T)$, which would make $T' + E$ a minimum spanning tree of $G$. If $T' + E \neq T$, then this is a contradiction and $T'$ cannot have less weight than $T - E$, so the tree does not need to be recomputed.

\section{}

\subsection{a}

There will be 2 ARP frames generated in this case, as follows:

\begin{tabular}{| c | r | r | r | r |}
    \hline
    \textbf{Frame Type} & \textbf{Sender IP} & \textbf{Sender MAC} & \textbf{Destination IP} & \textbf{Destination MAC} \\ \hline
    Request & 1.2.3.4 & MAC address of 1.2.3.4 & 1.2.3.10 & ff:ff:ff:ff:ff:ff \\ \hline
    Response & 1.2.3.10 & MAC address of 1.2.3.10 & 1.2.3.4 & MAC address of 1.2.3.4 \\ \hline
\end{tabular}

\subsection{b}

There will be a total of 6 ARP frames generated.

\begin{tabular}{| c | r | r | r | r |}
    \hline
    \textbf{Frame Type} & \textbf{Sender IP} & \textbf{Sender MAC} & \textbf{Destination IP} & \textbf{Destination MAC} \\ \hline
    Request & 1.2.3.4 & MAC address of 1.2.3.4 & 1.2.1.1 & ff:ff:ff:ff:ff:ff \\ \hline
    Response & 1.2.1.1 & MAC address of 1.2.1.1 & 1.2.3.4 & MAC address of 1.2.3.4 \\ \hline
    Request & 10.11.12.1 & MAC address of 10.11.12.1 & 10.11.12.25 & ff:ff:ff:ff:ff:ff \\ \hline
    Response & 10.11.12.15 & MAC address of 10.11.12.25 & 10.11.12.1 & MAC address of 10.11.12.1 \\ \hline
    Request & 15.16.17.25 & MAC address of 15.16.17.25 & 15.16.17.18 & ff:ff:ff:ff:ff:ff \\ \hline
    Response & 15.16.17.18 & MAC address of 15.16.17.18 & 15.16.17.25 & MAC address of 15.16.17.25 \\ \hline 
\end{tabular}

\end{document}
