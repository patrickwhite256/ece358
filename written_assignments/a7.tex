\documentclass[10pt,a4paper]{article}
\usepackage[utf8]{inputenc}
\usepackage{amsmath}
\usepackage{amsfonts}
\usepackage{booktabs}
\usepackage[margin=1in]{geometry}
\usepackage{amssymb}
\usepackage{tikz}
\usepackage{enumitem}
\author{Joel Mizzoni (jmizzoni), Patrick White (ps2white)}
\begin{document}
\title{ECE358 S'16 Assignment 7}
\maketitle
\section{}
In which we answer question 1
\section{}
In which we answer question 2
\section{}
Go-Back-N can suffer the same problem as selective repeat, but only if the size of the sender window is equal to the size of the sequence number space (which would be a bad decision). This would require the receiver to successfully receive and ACK a number of packets equal to the size of the sequence number space and for all of those ACKs to be subsequently lost.

\section{}

\begin{enumerate}[label=\alph*)]
    \item
        We avoid counting retransmissions because it might be the case that we timed out while an ACK was on its way to us and it is received very shortly after the we retransmit. We can't differentiate between an ACK that responded to the original transmission as opposed to a retransmission, so we just ignore measurements from retransmissions altogether.
    \item
        Fast retransmit waits for 3 duplicate ACKs because 1 or 2 duplicate ACKs is more likely to be a symptom of a reordering of packets rather than a loss. Once the third duplicate ACK is received, we can be fairly confident that the packet was lost [source: https://tools.ietf.org/html/rfc2001 Section 3: Fast Retransmit].
\end{enumerate}

\section{}
In which we answer question 5
\section{}
In which we answer question 6
\end{document}
