\documentclass[10pt,a4paper]{article}
\usepackage[utf8]{inputenc}
\usepackage{amsmath}
\usepackage{amsfonts}
\usepackage{booktabs}
\usepackage[margin=1in]{geometry}
\usepackage{amssymb}
\usepackage{tikz}
\usepackage{enumitem}
\author{Joel Mizzoni (jmizzoni), Patrick White (ps2white)}
\begin{document}
\title{ECE358 S'16 Assignment 7}
\maketitle
\section{}
We can assert the following: if \texttt{udt\_send} has a chance of sending an intact packet that is received by the host on the other side, rdt3.0 guarantees that a segment send by the sender will be received by the receiver in a finite amount of time.
\section{}
Yes, GBN correctly deals with reordered packets, as a corollary of the property that there is only one packet it will accept at a time.
If a packet with sequence number $n$ is expected by the receiver, but held by the network, the receiver will receive packets with sequence numbers greater than $n$.
Upon receiving each of these, the receiver will send an ACK for packet $n-1$.
When packet $n$ is eventually delivered, the receiver will ACK it and expect packet $n+1$.
The sender may receive the $n-1$ ACKs and resend packets starting with packet $n$.
This can result in the receiver receiving multiple copies of packet $n$, but this is not a problem, as the receiver will only deliver one of them since when the second one arrives, it will no longer be the expected packet.
\section{}
Go-Back-N can suffer the same problem as selective repeat, but only if the size of the sender window is equal to the size of the sequence number space (which would be a bad decision). This would require the receiver to successfully receive and ACK a number of packets equal to the size of the sequence number space and for all of those ACKs to be subsequently lost.

\section{}

\begin{enumerate}[label=\alph*)]
    \item
        We avoid counting retransmissions because it might be the case that we timed out while an ACK was on its way to us and it is received very shortly after the we retransmit. We can't differentiate between an ACK that responded to the original transmission as opposed to a retransmission, so we just ignore measurements from retransmissions altogether.
    \item
        Fast retransmit waits for 3 duplicate ACKs because 1 or 2 duplicate ACKs is more likely to be a symptom of a reordering of packets rather than a loss. Once the third duplicate ACK is received, we can be fairly confident that the packet was lost [source: https://tools.ietf.org/html/rfc2001 Section 3: Fast Retransmit].
\end{enumerate}

\section{}

Waiting for the final ACK from the client ensures that the server is aware of whether or not the client is in the ESTAB state, so it knows that there will be no duplicate connection requests that may be queued for later to cause the problems illustrated on slide 3-79.

\section{}
We begin by finding the number of bytes sent in between losses, $B$:
\begin{align*}
    B&=\frac{W}{2} + \left(\frac{W}{2} + MSS\right) + \left(\frac{W}{2} + 2MSS\right) + \cdots + W \\
     &=\frac{\left(\frac{W}{2MSS}+1\right)\left(\frac{W}{2}+W\right)}{2} \\
     &=\frac{\left(\frac{W+2\cdot MSS}{2MSS}\right)\left(\frac{3W}{2}\right)}{2} \\
     &=\frac{3W^2+6WMSS}{8MSS} \\
     &=\frac{3W^2}{8MSS} + 0.75 W \\
     &\approx \frac{3W^2}{8MSS}
\end{align*}
The number of segments sent in between losses, $S=B/MSS\approx\displaystyle\frac{3W^2}{8MSS^2}$.

The loss rate $L=1/S\approx\displaystyle\frac{8MSS^2}{3W^2}$.
\begin{align*}
    L&=\frac{8MSS^2}{3W^2} \\
    W^2&=\frac{8MSS^2}{3L} \\
    W&=MSS\sqrt{\frac{8}{3}}\frac{1}{\sqrt{L}} \\
    T&=\frac{3W}{4RTT} \\
    &\approx\frac{3}{4RTT}\cdot MSS\sqrt{\frac{8}{3}}\frac{1}{\sqrt{L}} \\
    &=\frac{3}{4}\sqrt{\frac{8}{3}}\frac{MSS}{RTT\sqrt{L}} \\
    &=\frac{1.22MSS}{RTT\sqrt{L}}
\end{align*}
\end{document}
