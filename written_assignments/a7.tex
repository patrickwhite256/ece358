\documentclass[10pt,a4paper]{article}
\usepackage[utf8]{inputenc}
\usepackage{amsmath}
\usepackage{amsfonts}
\usepackage{booktabs}
\usepackage[margin=1in]{geometry}
\usepackage{amssymb}
\usepackage{tikz}
\author{Joel Mizzoni (jmizzoni), Patrick White (ps2white)}
\begin{document}
\title{ECE358 S'16 Assignment 7}
\maketitle
\section{}
We can assert the following: if \texttt{udt\_send} has a chance of sending an intact packet that is received by the host on the other side, rdt3.0 guarantees that a segment send by the sender will be received by the receiver in a finite amount of time.
\section{}
Yes, GBN correctly deals with reordered packets, as a corollary of the property that there is only one packet it will accept at a time.
If a packet with sequence number $n$ is expected by the receiver, but held by the network, the receiver will receive packets with sequence numbers greater than $n$.
Upon receiving each of these, the receiver will send an ACK for packet $n-1$.
When packet $n$ is eventually delivered, the receiver will ACK it and expect packet $n+1$.
The sender may receive the $n-1$ ACKs and resend packets starting with packet $n$.
This can result in the receiver receiving multiple copies of packet $n$, but this is not a problem, as the receiver will only deliver one of them since when the second one arrives, it will no longer be the expected packet.
\section{}
In which we answer question three
\section{}
In which we answer question 4
\section{}
In which we answer question 5
\section{}
In which we answer question 6
\end{document}
